\documentclass[11pt,a5paper]{book}
\usepackage[utf8]{inputenc}
\usepackage[spanish]{babel}
\usepackage{geometry}
\usepackage{titlesec}
\usepackage{fancyhdr}
\usepackage{parskip}
\usepackage{graphicx}
\graphicspath{ {images/} }
\usepackage{needspace}

% Geometry
\geometry{
    top=2cm,
    bottom=2cm,
    left=2cm,
    right=2cm
}

% Styling
\titleformat{\chapter}[display]
  {\normalfont\huge\bfseries}{\chaptertitlename\ \thechapter}{20pt}{\Huge}

% Header/Footer
\pagestyle{fancy}
\fancyhf{}
\fancyhead[LE,RO]{\thepage}
\fancyhead[RE]{\textit{Mi amigo Jesucristo}}
\fancyhead[LO]{\textit{\leftmark}}

\title{Mi amigo Jesucristo}
\author{Francisco J. Bañezares}
\date{\today}

\begin{document}

\maketitle
\tableofcontents

\chapter*{Introducción: Entre la Partícula y la Vibración}

\begin{figure}[h]
    \centering
    \includegraphics[width=0.8\textwidth]{images/intro_quantum_luther.png}
\end{figure}

\addcontentsline{toc}{chapter}{Introducción: Entre la Partícula y la Vibración}

\needspace{5\baselineskip}
\section*{I. La Inquietud del Alma}

Desde que tengo uso de razón, he sentido una especie de vibración disonante cuando me acercaba a los grandes templos de piedra. No era falta de fe; al contrario, siempre he intuido que detrás del velo de la realidad existe una Fuente inagotable de amor. Mi conflicto, mi tormento silencioso, nacía del contraste entre esa intuición de un Dios infinito y la pequeñez de las cajas en las que intentábamos meterlo.

Crecí en el seno del catolicismo, respirando el incienso y admirando la liturgia, pero muy pronto, esa belleza se vio empañada por una sensación incómoda: la soberbia. Me dolía en el alma escuchar, implícita o explícitamente, que nosotros teníamos "la verdad" y los demás no. ¿Cómo podía ser que el Creador de un universo de cien mil millones de galaxias tuviera preferencia por un código postal espiritual concreto? Sentía que pecábamos de una arrogancia terrible al no reconocer la salvación para otros credos, al mirar con lástima o superioridad a quien buscaba la luz por otro camino. Esa exclusividad me parecía, paradójicamente, el acto más irreligioso de todos: limitar la misericordia de Dios a nuestras propias fronteras humanas.

\needspace{5\baselineskip}
\section*{II. Las Sombras en los Centros de Poder}

Pero mi tormento iba más allá de la teología. Al estudiar la historia, me topaba con muros manchados de sangre. La Inquisición, las cruzadas, las excomuniones... No podía evitar pensar: ¿Cómo hemos llegado de "ama a tu prójimo" a "quémalo si no piensa como tú"?

Durante años, esto me alejó. Sentía rabia. Veía la institución y solo veía las sombras. Sin embargo, con el tiempo y la madurez, entendí algo fundamental: la Iglesia, como cualquier estructura humana, está compuesta de personas. Y las personas somos antenas. Cuando el miedo se instala en los centros de poder, cuando la necesidad de control supera a la necesidad de servir, la vibración colectiva desciende.

La Inquisición no fue obra de Dios, ni siquiera fue obra de "la religión" en abstracto. Fue la consecuencia inevitable de una \textbf{baja vibración} instalada en la cúpula. El miedo a perder el poder, el miedo a lo diferente, el odio disfrazado de celo... todo eso son frecuencias densas, pesadas. Y cuando esa densidad se apodera de una jerarquía, el resultado es el sufrimiento. Obviamente, es algo que el propio Cristo, mi amigo Jesús, jamás hubiera permitido. Él, que detuvo las piedras contra la adúltera, jamás hubiera encendido una hoguera.

Entender esto me permitió perdonar. Entendí que no era "la Iglesia" la que fallaba, sino la vibración de los hombres que, en momentos oscuros, la dirigían. Y que, por debajo de esa cúpula de poder, siempre hubo miles de curas, monjas y laicos vibrando alto, dando de comer al hambriento, consolando al triste, manteniendo encendida la luz a pesar de la oscuridad de sus líderes.

\needspace{5\baselineskip}
\section*{III. La Física Cuántica del Espíritu}

Mi reconciliación definitiva llegó de la mano de la ciencia. Siempre me ha fascinado la física cuántica, esa rama del saber que nos dice que la realidad no es tan sólida como parece. Nos enseña que toda partícula subatómica es, a la vez, materia y onda. Es algo concreto y, al mismo tiempo, es pura vibración, pura posibilidad.

Ahí fue donde todo hizo clic. El Padre Nuestro, las enseñanzas de Jesús... no eran normas morales rígidas, ¡eran instrucciones de física cuántica!

Cuando Jesús nos dice "no temáis", no nos está dando una orden psicológica, nos está diciendo: "No bajéis vuestra frecuencia". El miedo es una vibración lenta, densa, que contrae la realidad. El amor, la paz, la seguridad... son vibraciones rápidas, expansivas, que crean luz.

Entendí que "caer en la tentación" no es comerse un pastel en cuaresma. Es la tentación de dejarse arrastrar por la gravedad de las bajas vibraciones: el odio, la venganza, la envidia, la soberbia. Cuando odias, te densificas. Te conviertes en "partícula" pesada, aislada, desconectada del todo. Cuando amas, te conviertes en "onda", te expandes, te conectas con el campo cuántico universal, con el Padre.

\needspace{5\baselineskip}
\section*{IV. Un Café con Lutero}

En este viaje de comprensión, a menudo me he imaginado tomando un café con Martín Lutero en el siglo XVI. Creo que habríamos sido buenos amigos. Él vio esa misma soberbia que a mí me atormentaba. Vio cómo la estructura se había vuelto tan densa, tan preocupada por vender bulas y construir basílicas de mármol, que se había olvidado de la vibración del Evangelio.

Lutero tuvo la valentía de decir: "La salvación no está en Roma, está en tu fe". Y yo añadiría, desde mi perspectiva del siglo XXI: "La salvación no depende de seguir al Papa, sino de la frecuencia de tu corazón".

Si tus obras, si tu vida diaria, están hechas desde la alta vibración del amor, no cabe duda de que estás salvado. Porque "estar salvado" no es un ticket para entrar en un club VIP después de morir. Estar salvado es vivir, aquí y ahora, en la frecuencia del Paraíso. Es estar en sintonía con la Fuente.

Si un budista, un ateo o un cristiano vibran en el amor incondicional, están en la misma frecuencia. Están "en Dios". Y ninguna bula, ningún decreto, puede cambiar esa realidad física y espiritual. La salvación es una cuestión de resonancia, no de burocracia.

\needspace{5\baselineskip}
\section*{V. La Luz que Prevalece}

Por eso escribo este libro. No para atacar a la religión, sino para rescatar su esencia vibracional. Escribo desde el amor, habiendo sanado esa lástima que sentía por la soberbia institucional. He dejado de mirar la oscuridad de la Inquisición para mirar la luz de los místicos, de los santos anónimos, de la gente buena que, en nombre de esa misma fe, ha vibrado tan alto que ha cambiado el mundo.

Este libro es una invitación a caminar con Jesús, no como un juez severo, sino como un maestro de vibración. Un amigo que nos enseña a navegar entre Madrid, Las Rozas y Colmenarejo sin perder la sintonía. Un guía que nos recuerda, una y otra vez, que somos luz, que somos onda, y que nuestro único deber es mantener la frecuencia alta, pase lo que pase a nuestro alrededor.

Así que, querido lector, olvida por un momento los dogmas, las culpas y los miedos heredados. Abre tu mente a la posibilidad de que todo, absolutamente todo, es música. Y que tú tienes el instrumento en tus manos para tocar la melodía más hermosa.

Bienvenido a "Mi amigo Jesucristo".

\newpage
\chapter{El Encuentro en Madrid}

\begin{figure}[h]
    \centering
    \includegraphics[width=0.8\textwidth]{images/madrid.png}
\end{figure}


Era una tarde de primavera en Madrid. El sol se filtraba entre los edificios de la Gran Vía, creando juegos de luces y sombras que parecían danzar sobre el asfalto. Yo caminaba con la prisa habitual de quien vive en la ciudad, con la mente llena de ruido: correos por responder, facturas, preocupaciones... vibrando bajo, muy bajo.

De repente, en la plaza de Callao, lo vi. No llevaba túnica, ni sandalias, ni tenía un halo brillante sobre la cabeza. Llevaba unos vaqueros desgastados, una camiseta blanca y unas zapatillas cómodas. Estaba sentado en un banco, mirando a la gente pasar con una sonrisa tranquila, una sonrisa que parecía detener el tiempo.

—Hola, Francisco —dijo cuando pasé cerca.

Me detuve en seco. Nadie me llamaba por mi nombre completo en la calle, y menos un desconocido. Pero al mirarlo a los ojos, supe que no era un desconocido. Había una familiaridad ancestral en su mirada, una paz que desarmaba cualquier defensa.

—¿Jesús? —pregunté, sintiéndome un poco ridículo.

—El mismo —respondió él, dándome una palmada en el espacio vacío del banco a su lado—. Siéntate un rato. Estás vibrando en una frecuencia que me está dando dolor de cabeza, y eso que yo aguanto mucho.

Me senté, todavía aturdido.
—¿Vibrando? —repetí.

—Sí, vibrando. Todo es vibración, amigo mío. El miedo, la prisa, el enfado... son vibraciones densas, pesadas. Te anclan al suelo y no te dejan ver el cielo, aunque lo tengas encima.

Miré hacia arriba. El cielo de Madrid estaba de un azul intenso, precioso. No me había fijado hasta ese momento.

—He venido a pasar unos días contigo —continuó—. Vamos a dar una vuelta. Necesitas recordar cómo sintonizar la radio de tu alma. Estás escuchando pura estática.

Y así, en medio del bullicio de Madrid, comenzó la aventura más extraña y maravillosa de mi vida. No con milagros de convertir agua en vino, sino con el milagro de transformar mi ruido interno en música.

\newpage
\chapter{Paseo por Las Rozas}

\begin{figure}[h]
    \centering
    \includegraphics[width=0.8\textwidth]{images/las_rozas_park.png}
\end{figure}


Al día siguiente, decidimos alejarnos un poco del centro. Fuimos a Las Rozas. Jesús quería ver cómo vivía la gente en las afueras, en esas urbanizaciones donde parece que todo está ordenado pero donde a veces el caos se esconde detrás de las puertas blindadas.

Caminábamos por el parque París. Él se detenía a oler las flores, a tocar la corteza de los árboles.
—¿Ves esto? —me dijo, señalando un roble—. Él no se preocupa por si mañana lloverá o si hará sol. Simplemente es. Está en su centro. Vibra en la frecuencia de la vida.

—Es fácil para un árbol —repliqué—. No tiene hipoteca.

Jesús se rió a carcajadas. Una risa limpia, contagiosa.
—La hipoteca... el gran monstruo moderno. Mira, Francisco, el problema no es la deuda del banco. El problema es la deuda que crees tener con el futuro. Vives pagando por adelantado con tu ansiedad un tiempo que aún no existe.

Nos sentamos en el césped. Un perro vino corriendo y le lamió la mano. Él lo acarició con ternura.
—"Danos hoy nuestro pan de cada día" —murmuró—. La gente cree que hablo solo de comida. Hablo de la energía necesaria para el \textit{ahora}. Cuando te preocupas por el mes que viene, estás gastando la energía de hoy en un problema imaginario. Bajas tu vibración. Te debilitas.

—¿Y cómo se sube la vibración? —pregunté, sintiendo que empezaba a entender.

—Agradeciendo —dijo él, mirando el lago artificial—. El agradecimiento es el ascensor del alma. Si agradeces este sol, este aire, este momento... subes. Si te quejas de lo que te falta... bajas. Es física espiritual básica.

En Las Rozas, entre chalets y coches de alta gama, aprendí que la verdadera riqueza no estaba en lo que poseíamos, sino en la capacidad de disfrutar lo que ya estaba ahí, gratis, para todos.

\newpage
\chapter{La Vibración y el Silencio}

\begin{figure}[h]
    \centering
    \includegraphics[width=0.8\textwidth]{images/silence_vibration.png}
\end{figure}


Tomamos el autobús hacia la sierra. El paisaje iba cambiando, volviéndose más verde, más salvaje. Jesús miraba por la ventana como un niño en su primera excursión.

—El ruido —dijo de repente—. Es la gran enfermedad de este siglo. No el ruido de los coches, sino el ruido de las pantallas, de las notificaciones, de las opiniones constantes.

Llegamos a un punto intermedio antes de Colmenarejo. Nos bajamos en un camino de tierra.
—Vamos a practicar el silencio —propuso.

—¿Callarnos?

—No solo callar la boca. Callar la mente. Quiero que escuches la vibración del universo.

Caminamos en silencio durante una hora. Al principio, mi mente era una lavadora centrifugando pensamientos: "¿Qué hago aquí caminando con Jesucristo?", "¿Me habré vuelto loco?", "Tengo que llamar a mi madre...". Pero poco a poco, el ritmo de mis pasos y la respiración tranquila de mi acompañante me fueron calmando.

Nos detuvimos en un alto desde donde se veía el horizonte.
—Cierra los ojos —me susurró—. Siente.

Lo hice. Y por primera vez en mucho tiempo, no sentí miedo. Sentí una especie de zumbido suave, cálido. Como si el aire me abrazara.

—Eso es —dijo él, como si pudiera leer mi sensación—. Eso es vibrar alto. Es la frecuencia del Amor. No el amor romántico de las películas, sino el Amor que sostiene los átomos unidos. Cuando estás ahí, nada puede tocarte. "Líbranos del mal" no es pedir que no existan los malos. Es pedir estar en una frecuencia donde la maldad no te alcance, donde no resuene contigo.

Abrí los ojos. El mundo parecía más brillante.
—Si vibras alto, el bajo no te ve —concluyó—. Es como si fueras invisible para la oscuridad.

\newpage
\chapter{Colmenarejo y la Naturaleza}

\begin{figure}[h]
    \centering
    \includegraphics[width=0.8\textwidth]{images/nature.png}
\end{figure}


Llegamos a Colmenarejo al atardecer. El pueblo tenía ese encanto de lo que está a medio camino entre la urbe y el campo. Fuimos hacia la zona de la Universidad, donde hay grandes espacios abiertos y se respira otro aire.

—Aquí se siente mejor —dijo Jesús, inspirando profundamente—. La naturaleza es la gran maestra de la vibración. Ella nunca se equivoca. Un pino no intenta ser una encina. Un río no intenta fluir hacia arriba.

Nos sentamos cerca de la ermita. El cielo se teñía de naranjas y violetas.
—Padre Nuestro que estás en los cielos... —empezó a recitar suavemente—. ¿Sabes dónde está el cielo, Francisco?

—¿Arriba? —señalé.

Él negó con la cabeza sonriendo.
—El cielo es un estado de conciencia. Es ese lugar interno donde todo está bien, donde hay paz absoluta. "Que estás en los cielos" significa que la Fuente de todo reside en esa alta vibración. Y tú tienes un acceso directo ahí. Tienes wifi directo con el Creador, pero a veces se te olvida la contraseña.

—¿Cuál es la contraseña?

—La rendición. Dejar de luchar contra lo que es. Aceptar. Fluir. Cuando te rindes a la vida, dejas de nadar contracorriente. Y entonces, la corriente te lleva. Y la corriente de la vida siempre te lleva hacia el bien, hacia el mar.

En Colmenarejo, bajo las primeras estrellas, entendí que rezar no era pedir cosas como quien hace la lista de la compra. Rezar era sintonizar. Era volver a poner el dial en la frecuencia correcta para escuchar la música que siempre está sonando, aunque nosotros estemos demasiado ocupados haciendo ruido para oírla.

\newpage
\chapter{El Padre Nuestro (Parte 1: Padre)}

\begin{figure}[h]
    \centering
    \includegraphics[width=0.8\textwidth]{images/cosmic_father.png}
\end{figure}


Regresamos a Madrid, pero esta vez fuimos al Retiro. Nos sentamos cerca del Palacio de Cristal.
—Hablemos de tu padre —dijo Jesús de repente.

Me tensé. Mi relación con mi padre no siempre había sido fácil.
—¿Por qué?

—Porque la oración empieza con "Padre Nuestro". Y mucha gente se atasca ahí. Si tu padre terrenal fue duro, o ausente, o crítico, proyectas esa imagen en Dios. Crees que Dios es un señor con barba que te vigila para castigarte si te equivocas.

Cogió una piedra del suelo y la lanzó al estanque. Las ondas se expandieron.
—La palabra "Padre" en arameo, \textit{Abwoon}, tiene un significado mucho más amplio. Es la Fuente, el Origen, la Respiración de la Vida. No es un señor. Es una Energía. Es la vibración original de la que todos salimos.

—Entonces, ¿no es un juez?

—¡Claro que no! —exclamó—. ¿Juzga el sol a las flores? ¿Les dice "tú sí floreces, te doy luz; tú no, te quedas a oscuras"? El sol brilla para todas. Dios es Amor incondicional. Si tú te cierras en una cueva, no es culpa del sol que estés a oscuras. Es tu elección.

—¿Y por qué "Nuestro"?

—Porque no es "Mío". No es propiedad privada de ninguna religión, ni de ninguna persona. Es de todos. Del cristiano, del budista, del ateo, del árbol y del perro. Todos venimos de la misma Fuente. Reconocer eso es el primer paso para vibrar alto: la Unidad. Si te ves separado de los demás, bajas tu vibración. Si te ves unido, subes.

\newpage
\chapter{El Padre Nuestro (Parte 2: Cielo)}

\begin{figure}[h]
    \centering
    \includegraphics[width=0.8\textwidth]{images/heaven_state.png}
\end{figure}


Caminando por la Gran Vía de nuevo, rodeados de anuncios luminosos y gente con bolsas de compras, Jesús señaló hacia arriba.
—"Que estás en los cielos". Ya te dije en Colmenarejo que el cielo es un estado de conciencia. Pero quiero que entiendas algo más.

Se detuvo frente a un escaparate de lujo.
—La gente busca el cielo aquí —señaló los relojes caros—. Creen que si tienen esto, serán felices. Que si consiguen ese trabajo, o esa pareja, tocarán el cielo. Pero el cielo no es un lugar al que vas, es un lugar desde el que vives.

—¿Cómo se vive desde el cielo aquí, en medio del tráfico y el estrés?

—Trayendo el cielo a la tierra. Esa es tu misión. "Venga a nosotros tu reino". No es esperar a morirse para ir a un lugar bonito con nubes. Es hacer que este lugar, aquí y ahora, se parezca a esa vibración de amor y paz.

—Eso suena difícil.

—Lo es si intentas cambiar el mundo de fuera sin cambiar el de dentro. Si tú estás en paz, si tú vibras en gratitud, creas una pequeña burbuja de cielo a tu alrededor. Y si muchos hacen eso, las burbujas se unen. Así es como viene el Reino. No con ejércitos, sino con corazones tranquilos.

Me miró fijamente.
—Tú eres un canal. Si estás sucio de miedo y rencor, la luz no pasa. Si te limpias, si vibras alto, el cielo pasa a través de ti y llega a la tierra. Eres un puente, Francisco. Todos lo sois.

\newpage
\chapter{El Pan de Cada Día}

\begin{figure}[h]
    \centering
    \includegraphics[width=0.8\textwidth]{images/bread_sparrows.png}
\end{figure}


Estábamos comiendo un bocadillo de calamares en la Plaza Mayor. Jesús lo disfrutaba como si fuera el manjar más exquisito del mundo.
—Esto es sagrado —dijo con la boca llena.

—¿El bocadillo?

—El acto de nutrirse. "Danos hoy nuestro pan de cada día".

Se limpió con una servilleta de papel.
—La gente vive con miedo a la escasez. Acumulan, guardan, temen perder. Eso es vibración de carencia. Cuando pides el pan de "hoy", estás confiando en que mañana también habrá.

—Pero hay que ser previsor...

—Ser previsor está bien. Vivir angustiado por el futuro es falta de fe en la Vida. Mira a los gorriones —señaló a unos pájaros que picoteaban migas—. No tienen graneros, y sin embargo, comen. La Vida sostiene a la Vida.

—¿Y qué pasa con la gente que no tiene para comer?

Su rostro se ensombreció un poco.
—Eso no es porque falte pan. Es porque sobra egoísmo. Si todos vibraran en la generosidad, si todos entendieran que somos uno, nadie pasaría hambre. El problema es la distribución, no la provisión. El Universo es abundante. La mente humana es la que crea la escasez.

Dio el último bocado.
—El "pan" también es la información, el conocimiento, el amor. Todo lo que te nutre. Pide lo que necesitas para hoy, para ser tu mejor versión hoy. No pidas para acumular y sentirte seguro. La única seguridad real es tu conexión con la Fuente.

\newpage
\chapter{Perdón y Deudas}

\begin{figure}[h]
    \centering
    \includegraphics[width=0.8\textwidth]{images/forgiveness_chains.png}
\end{figure}


Volvimos a Las Rozas, a una zona tranquila. Jesús quería hablar de algo pesado.
—"Perdona nuestras deudas, como también nosotros perdonamos a nuestros deudores".

—Esa es la parte difícil —admití—. Perdonar a quien te ha hecho daño...

—Es difícil porque lo ves como un favor que le haces al otro. "Yo, que soy tan bueno, te perdono a ti, que eres tan malo". Eso es ego. Eso no sirve.

—¿Entonces?

—El perdón es un acto de higiene personal. Es soltar un carbón ardiendo que tenías agarrado esperando tirárselo a otro. El único que se quema eres tú.

Se sentó en un banco de piedra.
—Cuando guardas rencor, vibras muy bajo. Te atas a esa persona y a ese momento doloroso con una cadena invisible. No puedes volar. No puedes avanzar. Perdonar es cortar la cadena. Es decir: "Te libero y me libero. Ya no te debo nada, y no me debes nada".

—¿Y si lo que hicieron fue muy grave?

—El dolor es inevitable, el sufrimiento es opcional. Si sigues repasando la ofensa en tu mente, la estás reviviendo una y otra vez. Te estás haciendo daño tú mismo ahora, no el otro. "Líbranos del mal" empieza por librarte del mal que tú mismo generas al no perdonar.

Me puso la mano en el hombro.
—Perdonar es vibrar tan alto que la ofensa ya no te alcanza. Es entender que el otro actuó desde su propia inconsciencia, desde su propio dolor. "No saben lo que hacen". Si supieran el daño que se hacen a sí mismos al dañar a otro, no lo harían.

\newpage
\chapter{No nos dejes caer en tentación (Vibrar Alto)}

\begin{figure}[h]
    \centering
    \includegraphics[width=0.8\textwidth]{images/vibration.png}
\end{figure}


Estábamos de nuevo en Colmenarejo, caminando por un sendero rodeado de jaras. El olor a campo era intenso.
—La tentación —dijo Jesús, arrancando una ramita de romero—. ¿Qué crees que es?

—¿El chocolate? ¿El dinero fácil?

—Eso son distracciones. La verdadera tentación es bajar tu vibración. Es la tentación de caer en el miedo, en la ira, en la desesperanza.

Se detuvo y me miró a los ojos.
—"No nos dejes caer en tentación" significa: "Ayúdame a mantenerme vibrando alto". Porque cuando vibras alto, estás en tu poder. Cuando caes, pierdes tu conexión.

—Pero a veces es inevitable enfadarse.

—Es inevitable sentir la emoción, sí. Pero es opcional quedarse a vivir en ella. La tentación es regodearse en el drama. Es llamar a tu amiga para contarle por décima vez lo mal que te trató tu jefe, solo para volver a sentir esa rabia. Eso es caer en la tentación. Es elegir el sufrimiento porque te hace sentir importante, víctima.

—¿Y cómo se evita?

—Con la consciencia. Cuando sientas que vas a caer, que vas a empezar a quejarte o a criticar, detente. Respira. Recuerda quién eres. Recuerda que eres luz. Y elige de nuevo. Elige el agradecimiento. Elige la esperanza. Elige el amor. Eso es vibrar alto. Y ahí, la tentación no tiene fuerza.

\newpage
\chapter{Líbranos del mal (Protección de Bajas Vibraciones)}

\begin{figure}[h]
    \centering
    \includegraphics[width=0.8\textwidth]{images/eagle_protection.png}
\end{figure}


La última noche en Madrid, fuimos a una terraza con vistas a la ciudad. Las luces parpadeaban abajo como un mar de estrellas eléctricas.
—Líbranos del mal —susurré, mirando el caos de tráfico.

—El mal no es un monstruo con cuernos —dijo Jesús, tomando un zumo de naranja—. El mal es simplemente la ausencia de luz. Es la inconsciencia. Es la gente que vibra tan bajo, tan denso, que solo puede proyectar dolor.

—¿Y cómo nos libramos de eso?

—No luchando contra ello. Si luchas contra la oscuridad, te manchas de oscuridad. Te libras elevándote por encima de ella.

Hizo un gesto con la mano, abarcando la ciudad.
—Imagina que eres un águila. Si hay una serpiente en el suelo, y bajas a pelear con ella, te puede morder. Pero si vuelas alto, la serpiente no te alcanza. "Líbranos del mal" es pedir las alas para volar alto, donde la vibración del odio, la envidia y el miedo no pueden tocarte.

—¿Y si alguien viene a atacarme?

—Si tú estás vibrando en amor puro, en paz absoluta, tu sola presencia puede desarmar al otro. O simplemente, la vida te moverá de lugar para que no te cruces con él. La sincronicidad te protege. Cuando vibras alto, te vuelves invisible para el radar del conflicto. Los que vibran bajo no te "ven", porque no resuenas en su frecuencia. Pasan de largo.

\newpage
\chapter{El Reino, el Poder y la Gloria}

\begin{figure}[h]
    \centering
    \includegraphics[width=0.8\textwidth]{images/kingdom_power.png}
\end{figure}


El viaje llegaba a su fin. Estábamos sentados en un banco de la Plaza de Oriente, frente al Palacio Real.
—Tuyo es el Reino, el Poder y la Gloria —dijo Jesús, mirando el imponente edificio—. Pero no ese reino de piedras y guardias.

—¿Cuál entonces?

—El Reino es tu paz interior. El Poder es tu capacidad de elegir tu vibración en cada momento, pase lo que pase fuera. Y la Gloria... la Gloria es la alegría de ser quien eres. De ser un hijo del Universo.

Se levantó y estiró los brazos, como abrazando el mundo.
—No busques el poder fuera. No busques que te aplaudan. El verdadero poder es silencioso. Es la certeza de que nunca estás solo, de que la Vida te sostiene. Cuando entiendes eso, cuando lo sientes en tus células, ya estás en el Reino. Aquí y ahora. Entre Madrid, Las Rozas y Colmenarejo. El Reino está donde tú estás, si tú estás despierto.

Me miró con una intensidad que me atravesó el alma.
—No olvides esto, Francisco. Tienes el poder. Úsalo para vibrar alto. Úsalo para iluminar tu rincón del mundo. Esa es tu única tarea.

\newpage
\chapter{Epílogo: Aventuras Finales y Despedida}

\begin{figure}[h]
    \centering
    \includegraphics[width=0.8\textwidth]{images/sunset_road.png}
\end{figure}


Jesús se despidió de mí en la estación de Atocha. Iba a coger un tren, dijo, hacia ninguna parte y hacia todas partes.
—¿Volveré a verte? —pregunté, sintiendo un nudo en la garganta.

—Siempre me ves —sonrió—. Estoy en la sonrisa del cajero del supermercado. En el viento que mueve los árboles de Colmenarejo. En el silencio de tu habitación. Pero sobre todo, estoy en ti. Cuando vibras alto, cuando amas, cuando perdonas... ahí estoy yo. Ahí somos uno.

Me dio un abrazo fuerte. Olía a madera y a lluvia fresca.
—No nos dejes caer en tentación —me susurró al oído—. Mantén la frecuencia, amigo. Mantén la música sonando. Y si alguna vez desafinas, no te juzgues. Simplemente, vuelve a sintonizar.

Se dio la vuelta y se mezcló con la multitud. Lo vi alejarse, con sus vaqueros y su caminar tranquilo, entre gente que corría mirando sus móviles. Y por un momento, me pareció ver que todo el vestíbulo de la estación brillaba con una luz dorada.

Salí a la calle. Madrid seguía igual: ruidosa, caótica, viva. Pero yo ya no era el mismo. Respiré hondo, sentí el sol en la cara y sonreí. Estaba vibrando alto. Y supe que, pasara lo que pasara, todo iba a estar bien.

\textbf{Fin.}

\newpage
\chapter*{Apéndice: Poema de Navidad}

\begin{figure}[h]
    \centering
    \includegraphics[width=0.8\textwidth]{images/christmas_star_modern.png}
\end{figure}

\addcontentsline{toc}{chapter}{Apéndice: Poema de Navidad}

\textbf{Navidad de Alta Frecuencia}

No busques al Niño en la paja fría de un tiempo remoto,\\
ni en el eco de un villancico que el viento ha roto.\\
No lo busques en la Gran Vía, bajo luces de neón,\\
donde el alma se vende barata en cada rincón.\\

Búscalo en el silencio que vibra en tu pecho,\\
en el templo sagrado que el Amor ha hecho.\\
Porque Jesús no lleva corona, ni oro, ni manto,\\
lleva vaqueros gastados y seca tu llanto.\\

Camina por Madrid, entre prisas y ruido,\\
un viajero eterno, por nadie temido.\\
Se sienta en Callao, te mira y sonríe:\\
"¿Por qué corres tanto? Párate y ríe."\\

"La Navidad no es fecha, ni cena, ni regalo,\\
es vibrar tan alto que olvides lo malo.\\
Es saber que el pesebre no es madera ni heno,\\
es tu propio corazón, de esperanza lleno."\\

Si vibras en miedo, si vibras en ira,\\
el Niño no nace, la estrella no brilla.\\
Pero si perdonas, si abrazas, si amas,\\
enciendes en ti las más puras llamas.\\

No nos dejes caer en la tentación de la tristeza,\\
levanta la vista, contempla la belleza.\\
Desde Las Rozas al cielo, de Colmenarejo al mar,\\
la única misión es aprender a Amar.\\

Líbranos del peso de vibrar en el suelo,\\
y danos las alas para alzar el vuelo.\\
Porque el Reino es ahora, la Gloria es hoy,\\
y en cada latido, contigo yo estoy.\\

Así que celebra, amigo, no el día marcado,\\
sino el Cristo viviente que tienes al lado.\\
Vibra alto, muy alto, que el mundo lo sienta,\\
que tu luz sea el faro en la gran tormenta.\\

Esta es la Navidad: no un cuento pasado,\\
sino el milagro eterno de haber despertado.\\

\newpage


\end{document}
